\documentclass[11pt,a4paper]{article}

\usepackage[utf8]{inputenc}
\usepackage[T1]{fontenc}
\usepackage{lmodern}
\usepackage{geometry}
\usepackage{graphicx}
\usepackage{booktabs}
\usepackage{array}
\usepackage{xcolor}
\usepackage{fancyhdr}
\usepackage{titlesec}
\usepackage{amsmath}
\usepackage{amssymb}

\geometry{margin=1in}

% Colors
\definecolor{kalshipurple}{RGB}{102, 51, 153}
\definecolor{bullgreen}{RGB}{34, 139, 34}
\definecolor{bearred}{RGB}{178, 34, 34}

% Header/Footer
\pagestyle{fancy}
\fancyhf{}
\fancyhead[L]{\textcolor{kalshipurple}{Kalshi Research Report}}
\fancyhead[R]{\thepage}
\fancyfoot[C]{\small Generated on January 27, 2026}

% Section formatting
\titleformat{\section}{\Large\bfseries\color{kalshipurple}}{\thesection}{1em}{}
\titleformat{\subsection}{\large\bfseries}{\thesubsection}{1em}{}

\begin{document}

% Title
\begin{center}
    {\LARGE\bfseries\textcolor{kalshipurple}{Market Research Report}}

    \vspace{0.5cm}

    {\Large Will Trump take back the Panama Canal?}

    \vspace{0.3cm}

    {\small Market Reference: \texttt{https://kalshi.com/markets/kxcanal/panama-canal-retaken/kxcanal-29}}
\end{center}

\vspace{1cm}

%% ============================================
\section{Market Overview}
%% ============================================

\begin{tabular}{@{}ll@{}}
    \textbf{Resolution Criteria:} & \parbox[t]{10cm}{If the United States government has taken control of at least some part of the Panama Canal before January 20, 2029, then the market resolves to Yes.} \\[0.5em]
    \textbf{Expiration:} & 2029-01-20T15:00:00Z \\
\end{tabular}

%% ============================================
\section{Market Pricing vs Independent Estimate}
%% ============================================

\begin{center}
\begin{tabular}{lcc}
    \toprule
    \textbf{Outcome} & \textbf{Market Price} & \textbf{Independent Estimate} \\
    \midrule
    \textcolor{bullgreen}{Yes} & 35.5\% & 5.0\% \\
    \textcolor{bearred}{No} & 64.5\% & 95.0\% \\
    \bottomrule
\end{tabular}
\end{center}

\vspace{0.5cm}

\textbf{Confidence Level:} High

%% ============================================
\section{Edge Analysis}
%% ============================================

There appears to be a substantial edge on the 'No' side. The 30+ percentage point gap suggests the market is heavily influenced by political speculation rather than sober analysis of legal realities. The Torrijos-Carter Treaties, international law, Panama's sovereignty, and massive diplomatic consequences create nearly insurmountable barriers. This could be a classic case of political betting markets overreacting to inflammatory rhetoric.

%% ============================================
\section{Research Summary}
%% ============================================

The United States transferred control of the Panama Canal to Panama in 1999 under the Torrijos-Carter Treaties signed in 1977. The canal is currently operated by the Panama Canal Authority (ACP), a Panamanian government agency. Any U.S. attempt to retake control would require either Panama's consent, military action, or some form of international legal process. Trump has made statements about the canal during his political career, including complaints about shipping fees and Chinese influence. The canal remains strategically and economically vital, handling about 6\% of global trade. Panama has consistently defended its sovereignty over the canal, and international law strongly supports Panama's ownership rights. Historical precedent shows that once decolonization or sovereignty transfers occur, reversing them through unilateral action is extremely rare and typically involves military force.

\subsection*{Sources}
\begin{itemize}
    \item Panama Canal Authority - Official website and historical documentation
    \item Torrijos-Carter Treaties (1977) - Treaty text and legal analysis
    \item Reuters, Associated Press - News coverage of recent Trump statements and Panama responses
    \item Congressional Research Service - Reports on Panama Canal and U.S.-Panama relations
    \item Maritime trade publications - Canal operations and shipping data
    \item International law journals - Analysis of sovereignty and treaty obligations
    \item U.S. State Department - Historical documents on canal transfer
\end{itemize}

%% ============================================
\section{Persona-Based Recommendations}
%% ============================================

\begin{center}
\begin{tabular}{|p{3cm}|p{3cm}|p{8cm}|}
    \hline
    \textbf{Persona} & \textbf{Position} & \textbf{Rationale} \\
    \hline
        Risk Averse & Moderate position on 'No' - Trump will NOT take back the Panama Canal & This aligns with risk-averse preferences for high-confidence plays with clear legal foundations. The... \\
        \hline
        Risk Seeking & Small contrarian position on 'Yes' - Trump WILL attempt to take back the canal & While the fundamental analysis suggests 'No' is correct, risk-seeking participants might find value ... \\
        \hline
        News Driven & Dynamic position starting with 'No', ready to pivot on Trump announcements & News-driven participants could benefit from the clear catalyst structure here. Any Trump statements,... \\
        \hline
        Macro Thinker & Strong 'No' position as part of broader 'anti-Trump rhetoric' portfolio & This fits into a macro framework where political rhetoric often diverges from implementable policy. ... \\
        \hline
        Casual Participant & Simple 'No' position with clear thesis: Treaties are binding & The thesis is memorably simple: 'International treaties signed by previous presidents cannot be unil... \\
        \hline
        Data Analyst & No position until better historical data on similar executive overreach & Data analysts would likely be frustrated by the lack of historical precedent for this exact scenario... \\
        \hline
\end{tabular}
\end{center}

%% ============================================
\section{Scenario Analysis}
%% ============================================

    \subsection*{Best Case}
    Trump's rhetoric remains purely political theater throughout his term. International law and diplomatic norms hold firm. Panama maintains strong sovereignty with broad international support. Any attempted pressure by the US faces unified opposition from Latin American allies, the EU, and China. Congressional Republicans distance themselves from any actual action beyond rhetoric. The market gradually recognizes this is empty posturing and 'No' probability rises to 85-90\%.

    \textbf{Probability Shift:} No probability increases from 64.5\% to 85-90\%

    \textbf{Key Triggers:}
    \begin{itemize}
            \item Congressional leaders explicitly reject canal action
            \item Strong international coalition supports Panama
            \item Trump focuses rhetoric on other issues
            \item Panama strengthens international partnerships
    \end{itemize}

    \subsection*{Worst Case}
    A major crisis in the canal (terrorism, Chinese military presence, or operational shutdown) provides Trump with a pretext for military action. He declares a national security emergency, bypasses normal diplomatic channels, and orders limited military occupation of critical canal infrastructure. This triggers a massive international crisis but technically satisfies the resolution criteria before legal challenges can be resolved.

    \textbf{Probability Shift:} Yes probability increases from 35.5\% to 60-70\%

    \textbf{Key Triggers:}
    \begin{itemize}
            \item Major security incident at the canal
            \item Evidence of foreign military presence
            \item Canal operations disrupted
            \item Trump declares national emergency
    \end{itemize}

    \subsection*{Most Likely}
    Trump continues canal rhetoric for domestic political consumption but takes no concrete action. Some economic pressure or trade negotiations may reference the canal, but no actual attempt to 'take control' occurs. International law and sovereignty norms prevent any serious challenge. The market slowly adjusts toward fundamentals but political betting maintains some premium above true probability.

    \textbf{Probability Shift:} No probability gradually increases to 75-80\%

    \textbf{Key Triggers:}
    \begin{itemize}
            \item Continued rhetoric without escalation
            \item Normal diplomatic relations maintained
            \item No emergency declarations
            \item Focus shifts to other political issues
    \end{itemize}


%% ============================================
\section*{Disclaimer}
%% ============================================

{\small\textit{This research report is for informational purposes only and does not constitute financial advice, investment advice, or a recommendation to buy or sell any securities or prediction market contracts. Prediction markets involve risk of loss. Past performance does not guarantee future results. Always do your own research and consider your own risk tolerance before participating in any market.}}

\end{document}
