\documentclass[11pt,a4paper]{article}

\usepackage[utf8]{inputenc}
\usepackage[T1]{fontenc}
\usepackage{lmodern}
\usepackage{geometry}
\usepackage{graphicx}
\usepackage{booktabs}
\usepackage{array}
\usepackage{xcolor}
\usepackage{fancyhdr}
\usepackage{titlesec}
\usepackage{amsmath}
\usepackage{amssymb}

\geometry{margin=1in}

% Colors
\definecolor{kalshipurple}{RGB}{102, 51, 153}
\definecolor{bullgreen}{RGB}{34, 139, 34}
\definecolor{bearred}{RGB}{178, 34, 34}

% Header/Footer
\pagestyle{fancy}
\fancyhf{}
\fancyhead[L]{\textcolor{kalshipurple}{Kalshi Research Report}}
\fancyhead[R]{\thepage}
\fancyfoot[C]{\small Generated on January 27, 2026}

% Section formatting
\titleformat{\section}{\Large\bfseries\color{kalshipurple}}{\thesection}{1em}{}
\titleformat{\subsection}{\large\bfseries}{\thesubsection}{1em}{}

\begin{document}

% Title
\begin{center}
    {\LARGE\bfseries\textcolor{kalshipurple}{Market Research Report}}

    \vspace{0.5cm}

    {\Large Will Saudi Arabia join the Abraham Accords by the end of 2027?}

    \vspace{0.3cm}

    {\small Market Reference: \texttt{https://kalshi.com/markets/kxabrahamsa/abraham-accord-saudi-arabia/kxabrahamsa-27}}
\end{center}

\vspace{1cm}

%% ============================================
\section{Market Overview}
%% ============================================

\begin{tabular}{@{}ll@{}}
    \textbf{Resolution Criteria:} & \parbox[t]{10cm}{UNKNOWN - Unable to access Kalshi API to retrieve exact resolution criteria. Would need to specify what constitutes 'joining the Abraham Accords' and the verification source.} \\[0.5em]
    \textbf{Expiration:} & End of 2027 (exact date unknown) \\
\end{tabular}

%% ============================================
\section{Market Pricing vs Independent Estimate}
%% ============================================

\begin{center}
\begin{tabular}{lcc}
    \toprule
    \textbf{Outcome} & \textbf{Market Price} & \textbf{Independent Estimate} \\
    \midrule
    \textcolor{bullgreen}{Yes} & N/A\% & 35.0\% \\
    \textcolor{bearred}{No} & N/A\% & 65.0\% \\
    \bottomrule
\end{tabular}
\end{center}

\vspace{0.5cm}

\textbf{Confidence Level:} Medium

%% ============================================
\section{Edge Analysis}
%% ============================================

Cannot identify trading edges without market pricing data. However, the 35\% probability for 'Yes' suggests this is viewed as a meaningful possibility despite current obstacles. If markets were pricing this significantly lower (say <20\%) due to recency bias from Gaza conflict, there could be value on the 'Yes' side. Conversely, if markets were pricing 'Yes' above 50\% based on pre-October 7 optimism, there could be value on the 'No' side.

%% ============================================
\section{Research Summary}
%% ============================================

The Abraham Accords, established in 2020, normalized relations between Israel and several Arab nations including the UAE, Bahrain, Morocco, and Sudan. Saudi Arabia's potential joining represents the most significant remaining prize for expanding the accords. Key factors include: (1) Saudi leadership under Crown Prince Mohammed bin Salman has shown openness to Israel normalization in exchange for major U.S. concessions including nuclear technology, defense guarantees, and Palestinian state progress; (2) The October 7, 2023 Hamas attack and subsequent Gaza conflict significantly complicated normalization efforts, with Saudi Arabia suspending talks and taking stronger pro-Palestinian stances; (3) Historical precedents show Saudi Arabia often follows regional trends after initial hesitation, having eventually accepted Egypt-Israel peace and maintained quiet cooperation with Israel on Iran containment; (4) Domestic Saudi considerations include managing conservative religious sentiment while pursuing Vision 2030 economic modernization goals that could benefit from Israeli technology partnerships.

\subsection*{Sources}
\begin{itemize}
    \item Wall Street Journal - September 2023 interview with Crown Prince Mohammed bin Salman on normalization progress
    \item Reuters - January 2024 reporting on Saudi Foreign Minister statements on Palestinian state requirements
    \item U.S. State Department - Official statements and briefings on Abraham Accords expansion efforts 2023-2024
    \item Abraham Accords text and framework documents from Trump and Biden administrations
    \item Council on Foreign Relations - Analysis of Middle East normalization trends and historical precedents
    \item Saudi Vision 2030 official documents outlining economic diversification goals
    \item Academic research on Arab-Israeli normalization patterns from Georgetown and Brookings institutions
    \item Diplomatic reporting from Financial Times, Associated Press on Saudi-Israel informal cooperation
\end{itemize}

%% ============================================
\section{Persona-Based Recommendations}
%% ============================================

\begin{center}
\begin{tabular}{|p{3cm}|p{3cm}|p{8cm}|}
    \hline
    \textbf{Persona} & \textbf{Position} & \textbf{Rationale} \\
    \hline
        Risk Averse & No position until market pricing is available & Without knowing market odds, impossible to identify the high-confidence edge this persona requires. ... \\
        \hline
        Risk Seeking & Small 'Yes' position if market prices below 25\% & The 35\% estimated probability suggests meaningful upside potential if markets are pricing this too p... \\
        \hline
        News Driven & Wait for specific catalysts before taking position & This market will likely move dramatically on news flow - ceasefire announcements, diplomatic visits,... \\
        \hline
        Macro Thinker & Consider as part of broader Middle East normalization portfolio & This connects to oil prices, US-Iran tensions, China's regional influence, and broader Arab-Israeli ... \\
        \hline
        Casual Participant & Small 'No' position if available around current estimate & Simple thesis: Gaza changed everything, Saudi public opinion matters, and 3+ years is long enough fo... \\
        \hline
        Data Analyst & No position - insufficient quantitative foundation & Without market pricing data, historical precedent analysis, or polling data from Saudi Arabia, this ... \\
        \hline
\end{tabular}
\end{center}

%% ============================================
\section{Scenario Analysis}
%% ============================================

    \subsection*{Best Case}
    Gaza conflict reaches sustainable ceasefire by mid-2025, with Palestinian state pathway outlined. Trump administration or successor pursues aggressive Middle East diplomacy with major economic incentives. China's Belt and Road loses momentum while U.S.-led alternative gains traction. Iran faces internal instability reducing regional influence. Saudi Vision 2030 requires Western technology partnerships that normalization facilitates. Regional Sunni coalition against Iran solidifies.

    \textbf{Probability Shift:} Yes probability increases to 60-70\%

    \textbf{Key Triggers:}
    \begin{itemize}
            \item Durable Gaza ceasefire with Palestinian state framework
            \item Major U.S. economic package for Saudi modernization
            \item Iranian regime destabilization
            \item Clear Chinese economic slowdown
            \item New Saudi Crown Prince or succession dynamics
    \end{itemize}

    \subsection*{Worst Case}
    Gaza conflict expands into broader regional war involving Hezbollah and Iran. Saudi domestic opinion becomes permanently hostile to Israel normalization. Iran successfully develops nuclear weapons, forcing Saudi focus on security over normalization. China's economic influence in region grows, reducing dependence on U.S. alliance. Palestinian issue remains unresolved with continued violence, making Saudi normalization politically impossible.

    \textbf{Probability Shift:} Yes probability drops to 10-15\%

    \textbf{Key Triggers:}
    \begin{itemize}
            \item Regional war involving Iran
            \item Iranian nuclear weapon
            \item Major Palestinian civilian casualties
            \item Collapse of U.S. Middle East influence
            \item Saudi leadership change opposing normalization
    \end{itemize}

    \subsection*{Most Likely}
    Gaza conflict reaches unstable ceasefire but Palestinian issue remains unresolved. Saudi Arabia continues private cooperation with Israel while avoiding public normalization. Economic interests drive deeper unofficial ties, but domestic and regional political constraints prevent formal Abraham Accords membership. Status quo of de facto cooperation without de jure recognition continues.

    \textbf{Probability Shift:} Yes probability remains around 30-35\%

    \textbf{Key Triggers:}
    \begin{itemize}
            \item Gaza ceasefire without Palestinian resolution
            \item Continued but limited U.S. engagement
            \item Gradual Iranian containment without regime change
            \item Steady but unspectacular Saudi economic diversification
    \end{itemize}


%% ============================================
\section*{Disclaimer}
%% ============================================

{\small\textit{This research report is for informational purposes only and does not constitute financial advice, investment advice, or a recommendation to buy or sell any securities or prediction market contracts. Prediction markets involve risk of loss. Past performance does not guarantee future results. Always do your own research and consider your own risk tolerance before participating in any market.}}

\end{document}
