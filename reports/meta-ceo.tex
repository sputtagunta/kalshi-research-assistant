\documentclass[11pt,a4paper]{article}

\usepackage[utf8]{inputenc}
\usepackage[T1]{fontenc}
\usepackage{lmodern}
\usepackage{geometry}
\usepackage{graphicx}
\usepackage{booktabs}
\usepackage{array}
\usepackage{xcolor}
\usepackage{fancyhdr}
\usepackage{titlesec}
\usepackage{amsmath}
\usepackage{amssymb}

\geometry{margin=1in}

% Colors
\definecolor{kalshipurple}{RGB}{102, 51, 153}
\definecolor{bullgreen}{RGB}{34, 139, 34}
\definecolor{bearred}{RGB}{178, 34, 34}

% Header/Footer
\pagestyle{fancy}
\fancyhf{}
\fancyhead[L]{\textcolor{kalshipurple}{Kalshi Research Report}}
\fancyhead[R]{\thepage}
\fancyfoot[C]{\small Generated on January 27, 2026}

% Section formatting
\titleformat{\section}{\Large\bfseries\color{kalshipurple}}{\thesection}{1em}{}
\titleformat{\subsection}{\large\bfseries}{\thesubsection}{1em}{}

\begin{document}

% Title
\begin{center}
    {\LARGE\bfseries\textcolor{kalshipurple}{Market Research Report}}

    \vspace{0.5cm}

    {\Large Will the CEO of Meta Platforms Inc leave in 2025?}

    \vspace{0.3cm}

    {\small Market Reference: \texttt{https://kalshi.com/markets/kxceometa/meta-ceo-leaving/kxceometa-25}}
\end{center}

\vspace{1cm}

%% ============================================
\section{Market Overview}
%% ============================================

\begin{tabular}{@{}ll@{}}
    \textbf{Resolution Criteria:} & \parbox[t]{10cm}{If the CEO of Meta Platforms Inc leaves before Jan 1, 2026, then the market resolves to Yes.

Additional terms: The CEO must actually leave their position for the market to resolve as Yes, rather than merely announcing it.} \\[0.5em]
    \textbf{Expiration:} & 2025-12-31T15:00:00Z \\
\end{tabular}

%% ============================================
\section{Market Pricing vs Independent Estimate}
%% ============================================

\begin{center}
\begin{tabular}{lcc}
    \toprule
    \textbf{Outcome} & \textbf{Market Price} & \textbf{Independent Estimate} \\
    \midrule
    \textcolor{bullgreen}{Yes} & 50.0\% & 5.0\% \\
    \textcolor{bearred}{No} & 50.0\% & 95.0\% \\
    \bottomrule
\end{tabular}
\end{center}

\vspace{0.5cm}

\textbf{Confidence Level:} High

%% ============================================
\section{Edge Analysis}
%% ============================================

Massive edge appears to exist favoring 'No' (Zuckerberg stays). The 45 percentage point gap suggests either: 1) Market participants don't understand Meta's governance structure, 2) This is a very illiquid/inefficient market with default 50/50 pricing, or 3) There's material non-public information. The structural factors (voting control, age, performance) strongly support the 'No' position being underpriced.

%% ============================================
\section{Research Summary}
%% ============================================

Mark Zuckerberg has been CEO of Meta Platforms Inc since founding Facebook in 2004, maintaining control through a dual-class share structure that gives him majority voting power despite owning a minority of economic shares. As of late 2024, Zuckerberg holds approximately 13\% of Meta's economic value but controls about 58\% of voting power through Class B shares. This structure makes it extremely difficult for external shareholders to force his departure. Historically, Zuckerberg has shown no indication of stepping down voluntarily, and at age 40, he is relatively young for tech CEO standards. Meta has faced various challenges including regulatory scrutiny, privacy concerns, and significant investments in the metaverse, but Zuckerberg has consistently maintained his leadership position through these periods. The company reported strong financial performance in 2024, with significant revenue growth and successful AI initiatives, which typically reduces pressure for CEO changes.

\subsection*{Sources}
\begin{itemize}
    \item SEC Form DEF 14A filings - Meta's proxy statements detailing share structure and voting control
    \item Meta Platforms Q3 2024 earnings report - Official financial performance data
    \item Bloomberg, Reuters, Wall Street Journal coverage of Meta corporate governance - Financial news reporting
    \item Meta Platforms investor relations website - Official company information
    \item Historical SEC filings tracking Zuckerberg's ownership and control over time
    \item Technology industry analysis from established financial publications
    \item Corporate governance research from institutional investors and proxy advisory firms
\end{itemize}

%% ============================================
\section{Persona-Based Recommendations}
%% ============================================

\begin{center}
\begin{tabular}{|p{3cm}|p{3cm}|p{8cm}|}
    \hline
    \textbf{Persona} & \textbf{Position} & \textbf{Rationale} \\
    \hline
        Risk Averse & Strong 'No' position (Zuckerberg stays) & This appears to offer an unusually high-confidence edge with structural protections. Zuckerberg's 58... \\
        \hline
        Risk Seeking & Contrarian 'Yes' position with small size & While the base case strongly favors Zuckerberg staying, a 50\% market price on departure creates asym... \\
        \hline
        News Driven & Monitor for 'No' entry points around earnings/events & This market likely moves on Meta-related news flow rather than fundamental analysis of departure pro... \\
        \hline
        Macro Thinker & Tactical 'No' position as tech leadership stability play & This position correlates with broader themes around founder-led tech stability and corporate governa... \\
        \hline
        Casual Participant & Simple 'No' position - Zuck owns the company & The thesis is straightforward: Zuckerberg controls 58\% of voting shares, meaning he can't be fired a... \\
        \hline
        Data Analyst & No position until better data on market mechanics & While the fundamental analysis strongly favors 'No', the 50/50 pricing is so disconnected from reali... \\
        \hline
\end{tabular}
\end{center}

%% ============================================
\section{Scenario Analysis}
%% ============================================

    \subsection*{Best Case}
    Zuckerberg remains CEO through 2025 with zero departure pressure. Meta continues strong financial performance, VR/AR initiatives show progress, and regulatory pressures remain manageable. His super-voting control remains unchallenged, and there are no personal or health issues. Market eventually recognizes the structural impossibility of forced departure and reprices to reflect reality.

    \textbf{Probability Shift:} Market probability for 'No' moves from 50\% to 85-90\% as participants understand Meta's governance structure and Zuckerberg's entrenchment

    \textbf{Key Triggers:}
    \begin{itemize}
            \item Strong Q1-Q3 2025 earnings reports
            \item Positive progress on Reality Labs/metaverse initiatives
            \item No regulatory actions targeting Zuckerberg personally
            \item Market education about dual-class share structure
    \end{itemize}

    \subsection*{Worst Case}
    Unexpected personal circumstances force Zuckerberg's departure despite his voting control. This could include serious health issues, family circumstances, or voluntary decision to step down for personal projects. Alternatively, extraordinary regulatory pressure or criminal charges could create unbearable pressure to resign voluntarily.

    \textbf{Probability Shift:} Market probability for 'Yes' jumps to 70-80\% once departure catalyst becomes apparent, but remains low (5-15\%) until such catalyst emerges

    \textbf{Key Triggers:}
    \begin{itemize}
            \item Serious health issue or family emergency
            \item Criminal charges or extreme regulatory pressure
            \item Voluntary announcement of desire to focus on philanthropy/other ventures
            \item Major scandal involving Zuckerberg personally
    \end{itemize}

    \subsection*{Most Likely}
    Zuckerberg remains CEO throughout 2025 with normal business volatility but no departure pressure. Meta performs adequately with mixed quarterly results, some regulatory challenges emerge but remain manageable, and VR/AR progress is incremental. No major personal or health issues arise. Market slowly becomes more efficient in pricing but never fully corrects the mispricing.

    \textbf{Probability Shift:} Market probability for 'No' gradually increases from 50\% to 65-75\% as 2025 progresses without departure signals, but significant mispricing persists

    \textbf{Key Triggers:}
    \begin{itemize}
            \item Mixed but acceptable quarterly earnings
            \item Routine regulatory scrutiny without major escalation
            \item Continued public appearances and engagement by Zuckerberg
            \item No succession planning announcements or structural changes
    \end{itemize}


%% ============================================
\section*{Disclaimer}
%% ============================================

{\small\textit{This research report is for informational purposes only and does not constitute financial advice, investment advice, or a recommendation to buy or sell any securities or prediction market contracts. Prediction markets involve risk of loss. Past performance does not guarantee future results. Always do your own research and consider your own risk tolerance before participating in any market.}}

\end{document}
