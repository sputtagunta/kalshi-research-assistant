\documentclass[11pt,a4paper]{article}

\usepackage[utf8]{inputenc}
\usepackage[T1]{fontenc}
\usepackage{lmodern}
\usepackage{geometry}
\usepackage{graphicx}
\usepackage{booktabs}
\usepackage{array}
\usepackage{xcolor}
\usepackage{fancyhdr}
\usepackage{titlesec}
\usepackage{amsmath}
\usepackage{amssymb}

\geometry{margin=1in}

% Colors
\definecolor{kalshipurple}{RGB}{102, 51, 153}
\definecolor{bullgreen}{RGB}{34, 139, 34}
\definecolor{bearred}{RGB}{178, 34, 34}

% Header/Footer
\pagestyle{fancy}
\fancyhf{}
\fancyhead[L]{\textcolor{kalshipurple}{Kalshi Research Report}}
\fancyhead[R]{\thepage}
\fancyfoot[C]{\small Generated on January 27, 2026}

% Section formatting
\titleformat{\section}{\Large\bfseries\color{kalshipurple}}{\thesection}{1em}{}
\titleformat{\subsection}{\large\bfseries}{\thesubsection}{1em}{}

\begin{document}

% Title
\begin{center}
    {\LARGE\bfseries\textcolor{kalshipurple}{Market Research Report}}

    \vspace{0.5cm}

    {\Large Will the IMF declare a global recession before 2027?}

    \vspace{0.3cm}

    {\small Market Reference: \texttt{https://kalshi.com/markets/kximfrecess/imf-global-recession/kximfrecess-27}}
\end{center}

\vspace{1cm}

%% ============================================
\section{Market Overview}
%% ============================================

\begin{tabular}{@{}ll@{}}
    \textbf{Resolution Criteria:} & \parbox[t]{10cm}{If the International Monetary Fund declares the world is in recession before 2027 in one of their World Economic Outlook reports, then the market resolves to Yes.} \\[0.5em]
    \textbf{Expiration:} & 2027-01-01T15:00:00Z \\
\end{tabular}

%% ============================================
\section{Market Pricing vs Independent Estimate}
%% ============================================

\begin{center}
\begin{tabular}{lcc}
    \toprule
    \textbf{Outcome} & \textbf{Market Price} & \textbf{Independent Estimate} \\
    \midrule
    \textcolor{bullgreen}{Yes} & 25.5\% & 25.0\% \\
    \textcolor{bearred}{No} & 74.5\% & 75.0\% \\
    \bottomrule
\end{tabular}
\end{center}

\vspace{0.5cm}

\textbf{Confidence Level:} Medium

%% ============================================
\section{Edge Analysis}
%% ============================================

No meaningful edge exists in this market. The convergence between independent analysis and market pricing is remarkably close (0.5\% difference), suggesting efficient price discovery. Any perceived edge would be overwhelmed by transaction costs, bid-ask spreads, and the inherent uncertainty in forecasting macro events 2-3 years out.

%% ============================================
\section{Research Summary}
%% ============================================

The IMF has historically declared global recessions very rarely, with only four instances since 1970: 1975, 1982, 1991, and 2009. The IMF defines a global recession as when global GDP growth falls below 3\%, typically accompanied by declining per capita GDP in many countries. The IMF's World Economic Outlook reports are published twice yearly (April and October) with periodic updates. Current economic conditions show mixed signals - while inflation has been declining in many developed economies, central banks maintain elevated interest rates, geopolitical tensions persist, and several major economies face structural challenges. The IMF's October 2024 outlook projected moderate global growth around 3.2\% for 2024-2025, but warned of downside risks including trade tensions, financial market volatility, and regional conflicts. Historical data suggests global recessions typically follow major financial crises, oil shocks, or synchronized monetary tightening across major economies.

\subsection*{Sources}
\begin{itemize}
    \item IMF World Economic Outlook Database - Historical global growth data and recession definitions
    \item IMF World Economic Outlook October 2024 - Latest global growth projections and risk assessment
    \item IMF Global Financial Stability Report 2024 - Analysis of financial sector vulnerabilities
    \item Federal Reserve Financial Stability Report May 2024 - U.S. financial system assessment
    \item IMF Article IV Consultation Reports 2024 - Country-specific economic assessments
    \item Bank for International Settlements Quarterly Review - Global financial market analysis
    \item OECD Economic Outlook 2024 - Comparative analysis of developed economy prospects
    \item World Bank Global Economic Prospects 2024 - Emerging market and development finance analysis
\end{itemize}

%% ============================================
\section{Persona-Based Recommendations}
%% ============================================

\begin{center}
\begin{tabular}{|p{3cm}|p{3cm}|p{8cm}|}
    \hline
    \textbf{Persona} & \textbf{Position} & \textbf{Rationale} \\
    \hline
        Risk Averse & No position - wait for better opportunities & With only a 0.5\% difference between market and estimated probabilities, there's no meaningful edge t... \\
        \hline
        Risk Seeking & Small 'Yes' position as portfolio diversifier & While there's no mathematical edge, a recession bet could provide valuable portfolio insurance durin... \\
        \hline
        News Driven & Monitor but avoid - insufficient catalysts & This market lacks clear near-term catalysts that could drive meaningful price movement. News-driven ... \\
        \hline
        Macro Thinker & Consider as correlation hedge in broader portfolio & Rather than seeking alpha, macro thinkers might view this as a hedge instrument. A 'Yes' position co... \\
        \hline
        Casual Participant & Skip this market entirely & This market offers no clear thesis or edge that would appeal to casual participants. The complexity ... \\
        \hline
        Data Analyst & Use as benchmark for market efficiency research & The remarkable pricing accuracy (0.5\% difference) makes this an excellent case study for market effi... \\
        \hline
\end{tabular}
\end{center}

%% ============================================
\section{Scenario Analysis}
%% ============================================

    \subsection*{Best Case}
    Multiple overlapping economic shocks create a synchronized global downturn by 2025. A combination of persistent inflation forcing aggressive monetary tightening, geopolitical tensions disrupting trade, and financial system stress from commercial real estate or corporate debt markets triggers a clear recession across major economies. The IMF, seeing coordinated GDP contractions and rising unemployment globally, formally declares a global recession in their April 2025 WEO.

    \textbf{Probability Shift:} YES probability increases to 45-50\%, driven by clear recessionary indicators across G7 economies and emerging markets simultaneously

    \textbf{Key Triggers:}
    \begin{itemize}
            \item Central bank policy rates above 6\% in US/EU
            \item China GDP growth below 3\%
            \item Global trade volume declining 3+ consecutive quarters
            \item Banking sector stress with major institution failures
            \item Oil price shock above \$120/barrel sustained
    \end{itemize}

    \subsection*{Worst Case}
    Economic resilience proves stronger than expected. Central banks achieve 'soft landings' with inflation returning to target without triggering widespread unemployment. Technological productivity gains, particularly from AI adoption, support growth. Geopolitical tensions remain contained without major trade disruptions. Even if individual regions face slowdowns, the global economy avoids the synchronized contraction needed for an IMF recession declaration.

    \textbf{Probability Shift:} YES probability drops to 10-15\%, as resilient growth and policy effectiveness prevent the coordinated global downturn required

    \textbf{Key Triggers:}
    \begin{itemize}
            \item US unemployment remaining below 5\%
            \item China maintaining 4\%+ growth
            \item European recession lasting <2 quarters
            \item Inflation reaching central bank targets without major employment costs
            \item No major financial system disruptions
    \end{itemize}

    \subsection*{Most Likely}
    Mixed global economic performance with regional variations. Some major economies experience technical recessions (2+ quarters of contraction) but not simultaneously. The US might have a mild recession in 2024-2025, while Europe faces prolonged stagnation, and China manages slow but positive growth. The IMF acknowledges 'global economic weakness' or 'synchronized slowdown' but stops short of declaring a formal global recession, using more nuanced language about regional variations.

    \textbf{Probability Shift:} YES probability remains near current levels (20-30\%), as the threshold for 'global recession' declaration proves high despite widespread economic weakness

    \textbf{Key Triggers:}
    \begin{itemize}
            \item Regional recessions but not globally synchronized
            \item IMF using terms like 'growth recession' or 'synchronized slowdown'
            \item Global growth falling to 1-2\% but remaining positive
            \item Mixed employment outcomes across regions
            \item Financial system stress contained to specific sectors
    \end{itemize}


%% ============================================
\section*{Disclaimer}
%% ============================================

{\small\textit{This research report is for informational purposes only and does not constitute financial advice, investment advice, or a recommendation to buy or sell any securities or prediction market contracts. Prediction markets involve risk of loss. Past performance does not guarantee future results. Always do your own research and consider your own risk tolerance before participating in any market.}}

\end{document}
