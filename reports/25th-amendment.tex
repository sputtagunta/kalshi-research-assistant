\documentclass[11pt,a4paper]{article}

\usepackage[utf8]{inputenc}
\usepackage[T1]{fontenc}
\usepackage{lmodern}
\usepackage{geometry}
\usepackage{graphicx}
\usepackage{booktabs}
\usepackage{array}
\usepackage{xcolor}
\usepackage{fancyhdr}
\usepackage{titlesec}
\usepackage{amsmath}
\usepackage{amssymb}

\geometry{margin=1in}

% Colors
\definecolor{kalshipurple}{RGB}{102, 51, 153}
\definecolor{bullgreen}{RGB}{34, 139, 34}
\definecolor{bearred}{RGB}{178, 34, 34}

% Header/Footer
\pagestyle{fancy}
\fancyhf{}
\fancyhead[L]{\textcolor{kalshipurple}{Kalshi Research Report}}
\fancyhead[R]{\thepage}
\fancyfoot[C]{\small Generated on January 27, 2026}

% Section formatting
\titleformat{\section}{\Large\bfseries\color{kalshipurple}}{\thesection}{1em}{}
\titleformat{\subsection}{\large\bfseries}{\thesubsection}{1em}{}

\begin{document}

% Title
\begin{center}
    {\LARGE\bfseries\textcolor{kalshipurple}{Market Research Report}}

    \vspace{0.5cm}

    {\Large Will the 25th Amendment be used during Trump's Presidency?}

    \vspace{0.3cm}

    {\small Market Reference: \texttt{https://kalshi.com/markets/kxamend25/25th-amendment/kxamend25-29}}
\end{center}

\vspace{1cm}

%% ============================================
\section{Market Overview}
%% ============================================

\begin{tabular}{@{}ll@{}}
    \textbf{Resolution Criteria:} & \parbox[t]{10cm}{If, as part of Section IV of the 25th Amendment, the Vice President (and a majority of the Cabinet or some other body designated by Congress) transmit to Congress a written declaration that the President is unable to discharge the powers and duties of his office before January 20, 2029, then the market resolves to Yes.} \\[0.5em]
    \textbf{Expiration:} & 2029-01-20T15:00:00Z \\
\end{tabular}

%% ============================================
\section{Market Pricing vs Independent Estimate}
%% ============================================

\begin{center}
\begin{tabular}{lcc}
    \toprule
    \textbf{Outcome} & \textbf{Market Price} & \textbf{Independent Estimate} \\
    \midrule
    \textcolor{bullgreen}{Yes} & 29.5\% & 8.0\% \\
    \textcolor{bearred}{No} & 70.5\% & 92.0\% \\
    \bottomrule
\end{tabular}
\end{center}

\vspace{0.5cm}

\textbf{Confidence Level:} Medium

%% ============================================
\section{Edge Analysis}
%% ============================================

Strong potential edge exists in fading the 'Yes' outcome. The market appears to be pricing in sensational scenarios without properly weighing the constitutional mechanisms involved. The 25th Amendment Section IV requires an almost impossible level of political coordination - the VP must lead a rebellion against their own administration, convince a Cabinet majority, then potentially face down 2/3 majorities in both houses of Congress. Even severe health crises have historically been managed through voluntary transfers (Section III) or informal arrangements. The market may be conflating general Trump uncertainty with this specific, extremely difficult constitutional process.

%% ============================================
\section{Research Summary}
%% ============================================

The 25th Amendment, Section IV provides a mechanism for the Vice President and Cabinet majority (or other Congressional body) to declare a President unable to discharge duties. This provision has never been successfully invoked in U.S. history, though it was considered during periods of presidential incapacitation. Historical precedent shows extreme reluctance to use this mechanism, with instances like Reagan's surgery in 1985 and Trump's COVID-19 treatment in 2020 handled through temporary voluntary transfers under Section III instead. The process requires extraordinary political consensus - not just from the VP and Cabinet majority, but ultimately a two-thirds vote in both houses of Congress if the President contests the declaration. Key factors that could influence usage include: presidential health incidents, significant cognitive concerns, major scandals affecting presidential function, or severe political crises where governing becomes impossible.

\subsection*{Sources}
\begin{itemize}
    \item U.S. Constitution, 25th Amendment - Primary legal text defining the process
    \item National Archives - Historical documentation of 25th Amendment usage
    \item Reagan Presidential Library - Documentation of Section III usage in 1985
    \item Bush Presidential Library - Records of voluntary transfers during medical procedures
    \item Washington Post, New York Times - Reporting on 25th Amendment discussions during Trump's first term
    \item White House Historical Association - Documentation of presidential succession protocols
    \item Congressional Research Service reports - Analysis of 25th Amendment procedures and history
    \item Academic legal scholarship - Constitutional law analysis of Section IV requirements
\end{itemize}

%% ============================================
\section{Persona-Based Recommendations}
%% ============================================

\begin{center}
\begin{tabular}{|p{3cm}|p{3cm}|p{8cm}|}
    \hline
    \textbf{Persona} & \textbf{Position} & \textbf{Rationale} \\
    \hline
        Risk Averse & Moderate 'No' position - betting against 25th Amendment usage & This fits conservative risk preferences perfectly. Historical precedent strongly favors 'No' (zero s... \\
        \hline
        Risk Seeking & Large 'No' position with potential to add on any 'Yes' spikes & The 21.5 percentage point edge represents exactly the kind of asymmetric opportunity that appeals to... \\
        \hline
        News Driven & Dynamic 'No' position with active monitoring strategy & This market will likely see significant volatility around Trump health news, Cabinet departures, or ... \\
        \hline
        Macro Thinker & Moderate 'No' position as portfolio stability anchor & This position offers negative correlation with general 'Trump chaos' bets. While others might be lon... \\
        \hline
        Casual Participant & Small 'No' position based on simple thesis & Simple story: The 25th Amendment has never been used this way, and it's really, really hard to do le... \\
        \hline
        Data Analyst & Calculated 'No' position sized according to Kelly criterion & The numbers are compelling: 21.5 percentage point edge is quantitatively significant. Zero historica... \\
        \hline
\end{tabular}
\end{center}

%% ============================================
\section{Scenario Analysis}
%% ============================================

    \subsection*{Best Case}
    Trump serves full term with 25th Amendment never seriously considered. Any health issues are handled through voluntary Section III transfers or informal arrangements. Cabinet remains loyal, VP maintains supporting role, and political opposition focuses on electoral rather than constitutional remedies. Market gradually reprices toward historical baseline of zero successful Section IV invocations.

    \textbf{Probability Shift:} Market probability for 'Yes' falls from 29.5\% to 5-10\% as initial speculation fades and constitutional reality sets in

    \textbf{Key Triggers:}
    \begin{itemize}
            \item Trump completes first year without major health incidents
            \item Cabinet stability with minimal turnover
            \item VP Vance demonstrates continued loyalty in public statements
            \item Congressional Republicans maintain unified support
    \end{itemize}

    \subsection*{Worst Case}
    Serious health crisis occurs that Trump refuses to acknowledge, creating constitutional standoff. VP and Cabinet majority begin private discussions about Section IV while Trump remains publicly defiant. However, the process still stalls due to political calculations - either Cabinet members resign rather than participate, or Congress fails to achieve required supermajorities. Market realizes even in crisis scenarios, the constitutional bar remains too high.

    \textbf{Probability Shift:} Market probability spikes temporarily to 40-50\% during crisis, then falls back to 15-20\% as political reality becomes clear

    \textbf{Key Triggers:}
    \begin{itemize}
            \item Major health incident with Trump refusing voluntary transfer
            \item VP begins consulting with Cabinet members privately
            \item Media speculation about 25th Amendment discussions
            \item Key Cabinet resignations rather than participation in process
            \item Congressional Republicans signal they won't support removal
    \end{itemize}

    \subsection*{Most Likely}
    Standard presidential term with periodic health concerns, political controversies, and media speculation about 25th Amendment. Each incident causes brief market volatility but no serious constitutional process emerges. Any health issues are managed through existing protocols. The extraordinary political consensus required for Section IV never materializes, even during difficult periods. Market slowly adjusts toward more realistic probability pricing.

    \textbf{Probability Shift:} Gradual decline in 'Yes' probability from 29.5\% to 12-18\% over 12-24 months as speculative premium diminishes

    \textbf{Key Triggers:}
    \begin{itemize}
            \item Periodic health scares or controversial statements generate media discussion
            \item Market volatility around each incident but no actual Cabinet action
            \item VP and Cabinet members publicly dismiss 25th Amendment speculation
            \item Political opposition focuses on 2026 midterms and 2028 elections instead
    \end{itemize}


%% ============================================
\section*{Disclaimer}
%% ============================================

{\small\textit{This research report is for informational purposes only and does not constitute financial advice, investment advice, or a recommendation to buy or sell any securities or prediction market contracts. Prediction markets involve risk of loss. Past performance does not guarantee future results. Always do your own research and consider your own risk tolerance before participating in any market.}}

\end{document}
