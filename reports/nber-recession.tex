\documentclass[11pt,a4paper]{article}

\usepackage[utf8]{inputenc}
\usepackage[T1]{fontenc}
\usepackage{lmodern}
\usepackage{geometry}
\usepackage{graphicx}
\usepackage{booktabs}
\usepackage{array}
\usepackage{xcolor}
\usepackage{fancyhdr}
\usepackage{titlesec}
\usepackage{amsmath}
\usepackage{amssymb}

\geometry{margin=1in}

% Colors
\definecolor{kalshipurple}{RGB}{102, 51, 153}
\definecolor{bullgreen}{RGB}{34, 139, 34}
\definecolor{bearred}{RGB}{178, 34, 34}

% Header/Footer
\pagestyle{fancy}
\fancyhf{}
\fancyhead[L]{\textcolor{kalshipurple}{Kalshi Research Report}}
\fancyhead[R]{\thepage}
\fancyfoot[C]{\small Generated on January 27, 2026}

% Section formatting
\titleformat{\section}{\Large\bfseries\color{kalshipurple}}{\thesection}{1em}{}
\titleformat{\subsection}{\large\bfseries}{\thesubsection}{1em}{}

\begin{document}

% Title
\begin{center}
    {\LARGE\bfseries\textcolor{kalshipurple}{Market Research Report}}

    \vspace{0.5cm}

    {\Large Will there be a recession in 2025?}

    \vspace{0.3cm}

    {\small Market Reference: \texttt{https://kalshi.com/markets/kxrecssnber/recession/recssnber-25}}
\end{center}

\vspace{1cm}

%% ============================================
\section{Market Overview}
%% ============================================

\begin{tabular}{@{}ll@{}}
    \textbf{Resolution Criteria:} & \parbox[t]{10cm}{If there are two consecutive quarters of negative GDP growth in 2024 or 2025, according to the Bureau of Economic Analysis, then the market resolves to Yes.

Additional terms: The market will close at the sooner of the occurrence of the event or 8:25 AM ET on the morning of the expected release of the Advance Estimate of 2025 Q4 GDP. The market will expire at the sooner of the occurrence of the event or the first 10:00 AM ET after the release of the Advance Estimate of 2025 Q4 GDP.} \\[0.5em]
    \textbf{Expiration:} & 2026-01-31T15:00:00Z \\
\end{tabular}

%% ============================================
\section{Market Pricing vs Independent Estimate}
%% ============================================

\begin{center}
\begin{tabular}{lcc}
    \toprule
    \textbf{Outcome} & \textbf{Market Price} & \textbf{Independent Estimate} \\
    \midrule
    \textcolor{bullgreen}{Yes} & 0.5\% & 25.0\% \\
    \textcolor{bearred}{No} & 99.5\% & 75.0\% \\
    \bottomrule
\end{tabular}
\end{center}

\vspace{0.5cm}

\textbf{Confidence Level:} Medium

%% ============================================
\section{Edge Analysis}
%% ============================================

Significant edge appears to exist on the 'Yes' side. The 24.5 percentage point difference is enormous and suggests the market is either severely mispricing recession risk or operating on information/definitions I'm missing. Even accounting for uncertainty in my estimates, the market's 0.5\% probability seems unreasonably low given: 1) Historical base rates, 2) Current restrictive monetary policy, 3) Decelerating growth indicators. The market may be treating this as a technical definition bet rather than an economic reality assessment.

%% ============================================
\section{Research Summary}
%% ============================================

Current U.S. economic conditions show a mixed picture for 2025 recession risk. The Federal Reserve has been gradually lowering interest rates from their peak, with the federal funds rate at 4.5-4.75\% as of December 2024, down from over 5.25\% earlier in the year. Recent GDP growth has remained positive but has shown some deceleration, with Q3 2024 annualized growth at 2.8\%. The labor market remains relatively strong with unemployment at 4.2\% as of November 2024, though job growth has slowed from earlier in the year. Inflation has cooled significantly from 2022-2023 peaks, with core PCE at 2.8\% in October 2024, approaching the Fed's 2\% target. However, several risk factors persist including elevated interest rates compared to the 2010s, ongoing geopolitical tensions, and potential policy uncertainties. Historical context shows that recessions are relatively common - the U.S. has experienced 12 recessions since 1945, averaging roughly one every 6-7 years, with the last recession occurring in 2020.

\subsection*{Sources}
\begin{itemize}
    \item Federal Reserve Economic Data (FRED) - St. Louis Fed database for economic indicators
    \item Bureau of Economic Analysis - Official GDP and inflation data
    \item Bureau of Labor Statistics - Employment and labor market data
    \item Federal Reserve meeting minutes and statements - Official Fed communications
    \item National Bureau of Economic Research - Official U.S. recession dating
    \item Congressional Budget Office economic projections and analysis
    \item Federal Reserve Bank regional surveys (Beige Book, regional Fed reports)
    \item Treasury yield curve data from Federal Reserve and Treasury Department
\end{itemize}

%% ============================================
\section{Persona-Based Recommendations}
%% ============================================

\begin{center}
\begin{tabular}{|p{3cm}|p{3cm}|p{8cm}|}
    \hline
    \textbf{Persona} & \textbf{Position} & \textbf{Rationale} \\
    \hline
        Risk Averse & Small 'Yes' position with clear exit strategy & The 24.5\% edge is substantial enough to overcome risk-averse tendencies. At 0.5\% market probability,... \\
        \hline
        Risk Seeking & Aggressive 'Yes' position leveraging maximum edge & This represents exactly the type of asymmetric opportunity risk-seekers love - paying 0.5 cents for ... \\
        \hline
        News Driven & Opportunistic 'Yes' position with catalyst monitoring & The extreme mispricing suggests the market isn't pricing in potential negative catalysts like credit... \\
        \hline
        Macro Thinker & Strategic 'Yes' position as portfolio hedge & Beyond the direct edge, this position serves as valuable portfolio insurance. If recession occurs, t... \\
        \hline
        Casual Participant & Moderate 'Yes' position with simple thesis & Simple story: Market thinks there's almost zero chance of recession, but history shows recessions ha... \\
        \hline
        Data Analyst & Cautious 'Yes' position pending definition clarity & The quantitative edge is compelling (24.5\% differential), but the extreme market pricing suggests mi... \\
        \hline
\end{tabular}
\end{center}

%% ============================================
\section{Scenario Analysis}
%% ============================================

    \subsection*{Best Case}
    The Federal Reserve's restrictive monetary policy triggers a credit crunch by Q2 2025, leading to sharp business investment decline and consumer spending pullback. Corporate earnings deteriorate rapidly as high borrowing costs squeeze leveraged companies. Unemployment rises from current lows as companies implement layoffs. Two consecutive quarters of GDP decline occur in Q2 and Q3 2025, with the BEA confirming the recession by late 2025.

    \textbf{Probability Shift:} Market probability moves from 0.5\% to 15-25\% as economic data deteriorates and recession becomes apparent

    \textbf{Key Triggers:}
    \begin{itemize}
            \item Fed maintains rates above 5\% through Q1 2025
            \item Credit spreads widen significantly (>200bps)
            \item Unemployment rises above 4.5\%
            \item Corporate earnings decline 15\%+ year-over-year
            \item Consumer confidence drops below 80
    \end{itemize}

    \subsection*{Worst Case}
    The economy demonstrates remarkable resilience despite high interest rates. Productivity gains from AI and technology offset monetary tightening effects. Labor markets remain tight but inflation continues declining toward 2\% target. The Fed begins cutting rates by mid-2025, providing economic support. GDP growth remains positive throughout 2024-2025, though at slower pace. No consecutive quarters of negative growth occur.

    \textbf{Probability Shift:} Market probability stays near current 0.5\% levels as economic resilience becomes apparent

    \textbf{Key Triggers:}
    \begin{itemize}
            \item Core PCE inflation drops to 2.5\% by Q1 2025
            \item Fed begins cutting rates by June 2025
            \item Productivity growth exceeds 2\% annually
            \item Unemployment remains below 4.2\%
            \item Consumer spending stays robust despite high rates
    \end{itemize}

    \subsection*{Most Likely}
    Economic growth significantly slows but narrowly avoids the technical definition of recession. One quarter of negative growth occurs (likely Q2 or Q3 2025) but is followed by flat to slightly positive growth, preventing two consecutive negative quarters. The economy experiences a 'growth recession' with rising unemployment and declining business activity, but GDP remains just positive in at least one quarter during any two-quarter period.

    \textbf{Probability Shift:} Market probability rises to 3-8\% as recession risks become apparent, but technical definition isn't met

    \textbf{Key Triggers:}
    \begin{itemize}
            \item GDP growth averages 0-1\% in 2025
            \item One quarter shows negative growth but recovery follows
            \item Unemployment rises to 4.5-5.0\%
            \item Corporate earnings flat to slightly negative
            \item Consumer spending growth slows to 0-1\%
    \end{itemize}


%% ============================================
\section*{Disclaimer}
%% ============================================

{\small\textit{This research report is for informational purposes only and does not constitute financial advice, investment advice, or a recommendation to buy or sell any securities or prediction market contracts. Prediction markets involve risk of loss. Past performance does not guarantee future results. Always do your own research and consider your own risk tolerance before participating in any market.}}

\end{document}
