\documentclass[11pt,a4paper]{article}

\usepackage[utf8]{inputenc}
\usepackage[T1]{fontenc}
\usepackage{lmodern}
\usepackage{geometry}
\usepackage{graphicx}
\usepackage{booktabs}
\usepackage{array}
\usepackage{xcolor}
\usepackage{fancyhdr}
\usepackage{titlesec}
\usepackage{amsmath}
\usepackage{amssymb}

\geometry{margin=1in}

% Colors
\definecolor{kalshipurple}{RGB}{102, 51, 153}
\definecolor{bullgreen}{RGB}{34, 139, 34}
\definecolor{bearred}{RGB}{178, 34, 34}

% Header/Footer
\pagestyle{fancy}
\fancyhf{}
\fancyhead[L]{\textcolor{kalshipurple}{Kalshi Research Report}}
\fancyhead[R]{\thepage}
\fancyfoot[C]{\small Generated on January 27, 2026}

% Section formatting
\titleformat{\section}{\Large\bfseries\color{kalshipurple}}{\thesection}{1em}{}
\titleformat{\subsection}{\large\bfseries}{\thesubsection}{1em}{}

\begin{document}

% Title
\begin{center}
    {\LARGE\bfseries\textcolor{kalshipurple}{Market Research Report}}

    \vspace{0.5cm}

    {\Large Will the US default on its debt by Dec 31, 2025??}

    \vspace{0.3cm}

    {\small Market Reference: \texttt{https://kalshi.com/markets/kxdefault/us-defaults/kxdefault-25dec31}}
\end{center}

\vspace{1cm}

%% ============================================
\section{Market Overview}
%% ============================================

\begin{tabular}{@{}ll@{}}
    \textbf{Resolution Criteria:} & \parbox[t]{10cm}{If the U.S. Department of the Treasury announces that the United States Federal Government failed to make a scheduled payment on a Treasury note, bond, or bill; or that one of the three major credit ratings agencies consider any United States debt in any form of default, before 2026, then the market resolves to Yes.} \\[0.5em]
    \textbf{Expiration:} & 2026-01-01T15:00:00Z \\
\end{tabular}

%% ============================================
\section{Market Pricing vs Independent Estimate}
%% ============================================

\begin{center}
\begin{tabular}{lcc}
    \toprule
    \textbf{Outcome} & \textbf{Market Price} & \textbf{Independent Estimate} \\
    \midrule
    \textcolor{bullgreen}{Yes} & 50.0\% & 15.0\% \\
    \textcolor{bearred}{No} & 50.0\% & 85.0\% \\
    \bottomrule
\end{tabular}
\end{center}

\vspace{0.5cm}

\textbf{Confidence Level:} Medium

%% ============================================
\section{Edge Analysis}
%% ============================================

Significant edge appears to exist in betting AGAINST default (No position). The 35 percentage point gap suggests market is dramatically overpricing tail risk. The 50/50 pricing feels like 'maximum uncertainty' pricing rather than fundamental analysis. Historical base rates strongly favor resolution, and the economic incentives remain overwhelming despite political dysfunction.

%% ============================================
\section{Research Summary}
%% ============================================

The US debt ceiling is a critical factor in potential default scenarios. As of late 2024, the debt ceiling was suspended until January 1, 2025, after which Treasury will need to use extraordinary measures to avoid breaching the limit. Historical precedent shows the US has never defaulted on its debt obligations, though several debt ceiling crises have brought the country close to technical default (notably 2011, 2013, and 2023). The Treasury's extraordinary measures typically provide several months of additional borrowing capacity, but the exact duration depends on government revenues and expenditures. Credit rating agencies have previously downgraded US debt during debt ceiling standoffs - S\&P downgraded the US from AAA to AA+ in August 2011 and maintained this rating through 2024. Moody's and Fitch have issued negative outlooks during debt ceiling crises but maintained their AAA ratings, though both have warned of potential downgrades if political dysfunction continues. The resolution criteria would be triggered either by an actual missed payment or a credit rating agency declaring any form of default.

\subsection*{Sources}
\begin{itemize}
    \item Congressional Budget Office - Debt ceiling analysis reports and extraordinary measures documentation
    \item US Treasury Department - Monthly treasury statements and debt ceiling historical information
    \item S\&P Global Ratings - US sovereign credit rating reports and methodology
    \item Moody's Investors Service - US government credit opinions and rating actions
    \item Fitch Ratings - US sovereign rating reports and surveillance notes
    \item Public Law 118-5 (Fiscal Responsibility Act of 2023) - Congressional legislation
    \item Government Accountability Office - Reports on debt ceiling processes and risks
    \item Federal Reserve Economic Data - Historical debt and fiscal data
\end{itemize}

%% ============================================
\section{Persona-Based Recommendations}
%% ============================================

\begin{center}
\begin{tabular}{|p{3cm}|p{3cm}|p{8cm}|}
    \hline
    \textbf{Persona} & \textbf{Position} & \textbf{Rationale} \\
    \hline
        Risk Averse & Moderate 'No' position (betting against default) & The 35 percentage point edge combined with strong historical precedent creates a high-confidence opp... \\
        \hline
        Risk Seeking & Small 'Yes' position as contrarian hedge & While the base case strongly favors resolution, the current political environment shows unprecedente... \\
        \hline
        News Driven & Dynamic positioning around debt ceiling negotiations & This market will be highly sensitive to political news flow, congressional statements, and Treasury ... \\
        \hline
        Macro Thinker & 'No' position as part of broader stability thesis & US debt default would trigger global financial system collapse, making it the ultimate 'too big to f... \\
        \hline
        Casual Participant & Simple 'No' position - bet America doesn't implode & This is straightforward: despite all the political drama, America has never defaulted on its debt an... \\
        \hline
        Data Analyst & Cautious 'No' position with quantified sizing & Historical base rate shows 100\% resolution record across dozens of debt ceiling episodes. The 35 per... \\
        \hline
\end{tabular}
\end{center}

%% ============================================
\section{Scenario Analysis}
%% ============================================

    \subsection*{Best Case}
    Political leaders recognize the catastrophic economic consequences early and negotiate a clean debt ceiling increase or suspension by mid-2025. Treasury maintains full operational capacity without needing extraordinary measures. Market confidence remains high throughout the process, and the political theater is minimal compared to previous cycles.

    \textbf{Probability Shift:} No resolution probability increases to 95\%+. Market recognizes this was never really in doubt once adults took control of negotiations.

    \textbf{Key Triggers:}
    \begin{itemize}
            \item Early bipartisan statements about avoiding default
            \item Treasury Secretary provides clear timeline and assurances
            \item Congressional leadership schedules vote well before X-date
            \item Business community applies coordinated pressure for clean resolution
    \end{itemize}

    \subsection*{Worst Case}
    Severe political dysfunction coincides with divided government and razor-thin margins. Negotiations deadlock past the X-date, Treasury exhausts extraordinary measures, and a technical default occurs on a scheduled payment. However, this lasts only 1-3 days before political reality forces resolution, but triggers rating agency downgrade.

    \textbf{Probability Shift:} Yes resolution probability could spike to 30-40\% during peak crisis, then crash back down once resolved

    \textbf{Key Triggers:}
    \begin{itemize}
            \item Hardline factions refuse to compromise on spending/revenue
            \item Miscalculation of Treasury's extraordinary measures timeline
            \item External crisis diverts political attention at critical moment
            \item Rating agencies issue ultimatum that forces political hands
    \end{itemize}

    \subsection*{Most Likely}
    Familiar debt ceiling theater plays out with brinksmanship extending to within weeks of X-date. Treasury employs extraordinary measures for several months while Congress negotiates. Last-minute deal emerges combining debt ceiling increase with modest spending reforms, similar to previous resolutions. Some market volatility but no actual default.

    \textbf{Probability Shift:} No resolution probability settles around 90-92\% once pattern recognition kicks in

    \textbf{Key Triggers:}
    \begin{itemize}
            \item Treasury announces extraordinary measures deployment
            \item Congressional negotiations begin showing progress under market pressure
            \item Business leaders and economists warn of consequences
            \item Polling shows public opposes default across party lines
    \end{itemize}


%% ============================================
\section*{Disclaimer}
%% ============================================

{\small\textit{This research report is for informational purposes only and does not constitute financial advice, investment advice, or a recommendation to buy or sell any securities or prediction market contracts. Prediction markets involve risk of loss. Past performance does not guarantee future results. Always do your own research and consider your own risk tolerance before participating in any market.}}

\end{document}
