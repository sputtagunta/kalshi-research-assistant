\documentclass[11pt,a4paper]{article}

\usepackage[utf8]{inputenc}
\usepackage[T1]{fontenc}
\usepackage{lmodern}
\usepackage{geometry}
\usepackage{graphicx}
\usepackage{booktabs}
\usepackage{array}
\usepackage{xcolor}
\usepackage{fancyhdr}
\usepackage{titlesec}
\usepackage{amsmath}
\usepackage{amssymb}

\geometry{margin=1in}

% Colors
\definecolor{kalshipurple}{RGB}{102, 51, 153}
\definecolor{bullgreen}{RGB}{34, 139, 34}
\definecolor{bearred}{RGB}{178, 34, 34}

% Header/Footer
\pagestyle{fancy}
\fancyhf{}
\fancyhead[L]{\textcolor{kalshipurple}{Kalshi Research Report}}
\fancyhead[R]{\thepage}
\fancyfoot[C]{\small Generated on January 27, 2026}

% Section formatting
\titleformat{\section}{\Large\bfseries\color{kalshipurple}}{\thesection}{1em}{}
\titleformat{\subsection}{\large\bfseries}{\thesubsection}{1em}{}

\begin{document}

% Title
\begin{center}
    {\LARGE\bfseries\textcolor{kalshipurple}{Market Research Report}}

    \vspace{0.5cm}

    {\Large Will Donald Trump issue any executive action on imposing new or increased tariffs on any country, provided the action specifies an effective date (even if such date occurs after expiration) during in January 2026?}

    \vspace{0.3cm}

    {\small Market Reference: \texttt{https://kalshi.com/markets/kxnewtariffs/new-tariffs/kxnewtariffs-26feb01}}
\end{center}

\vspace{1cm}

%% ============================================
\section{Market Overview}
%% ============================================

\begin{tabular}{@{}ll@{}}
    \textbf{Resolution Criteria:} & \parbox[t]{10cm}{If Donald Trump has taken any executive action imposing new or increased tariffs, provided the action specifies an effective date (even if such date occurs after expiration) after Issuance and before February 1, 2026, then the market resolves to Yes.

Additional terms: The qualifying action must be of the specific type designated (executive order, presidential memorandum, proclamation, directive, determination, or finding), be signed by the President personally during the specified time period, and explicitly address the topic in the document's operative provisions, title, or official White House summary. Actions must have legal or policy effect - ceremonial proclamations without policy impact do not qualify unless specifically included. Actions that only incidentally mention the topic, statements without formal action, actions by cabinet members, legislative proposals without executive action, and signing statements do not qualify. The action must be publicly announced or documented by a Source Agency before expiration.} \\[0.5em]
    \textbf{Expiration:} & 2026-02-01T15:00:00Z \\
\end{tabular}

%% ============================================
\section{Market Pricing vs Independent Estimate}
%% ============================================

\begin{center}
\begin{tabular}{lcc}
    \toprule
    \textbf{Outcome} & \textbf{Market Price} & \textbf{Independent Estimate} \\
    \midrule
    \textcolor{bullgreen}{Yes} & 33.5\% & 75.0\% \\
    \textcolor{bearred}{No} & 66.5\% & 25.0\% \\
    \bottomrule
\end{tabular}
\end{center}

\vspace{0.5cm}

\textbf{Confidence Level:} Medium

%% ============================================
\section{Edge Analysis}
%% ============================================

A substantial edge appears to exist on the 'Yes' outcome. The 41.5\% probability gap is well beyond noise levels and suggests the market may be systematically undervaluing Trump's likelihood of rapid tariff action. This could stem from: 1) Market participants overweighting transition period constraints, 2) Insufficient weighting of Trump's first-term precedents, 3) Underestimating the pre-existing legal infrastructure for tariff actions, or 4) General market skepticism about campaign promise implementation speed.

%% ============================================
\section{Research Summary}
%% ============================================

Donald Trump has a well-documented history of using executive actions to implement tariff policies during his first presidency (2017-2021), particularly through presidential proclamations under Section 232 of the Trade Expansion Act and Section 301 of the Trade Act. His campaign rhetoric for 2024 included extensive tariff proposals, suggesting he would likely continue this approach if elected. The market specifically asks about January 2026, which would be his first month in office if he wins the 2024 election and is inaugurated on January 20, 2025. Historically, new presidents often issue significant executive actions early in their tenure, and Trump's previous pattern shows he issued tariff-related proclamations within his first year. The legal framework for presidential tariff authority remains intact, including emergency powers under the International Emergency Economic Powers Act (IEEPA) and trade authority under various statutes. However, any tariffs would face potential legal challenges, congressional scrutiny, and economic considerations that might influence timing and scope.

\subsection*{Sources}
\begin{itemize}
    \item Federal Register archives - Presidential Proclamations 9704, 9705 (2018 steel/aluminum tariffs)
    \item Trade Expansion Act of 1962, Section 232 - Congressional authority for national security tariffs
    \item Trade Act of 1974, Section 301 - Authority for retaliatory trade measures
    \item International Emergency Economic Powers Act (50 U.S.C. 1701) - Emergency trade authorities
    \item Congressional Research Service reports on presidential trade authorities
    \item White House archives from 2017-2021 showing Trump's executive action patterns
    \item U.S. Trade Representative historical records on Section 301 investigations
    \item Federal Register publication requirements for executive actions
\end{itemize}

%% ============================================
\section{Persona-Based Recommendations}
%% ============================================

\begin{center}
\begin{tabular}{|p{3cm}|p{3cm}|p{8cm}|}
    \hline
    \textbf{Persona} & \textbf{Position} & \textbf{Rationale} \\
    \hline
        Risk Averse & Small 'Yes' position or no position & While the edge appears substantial (41.5\%), risk-averse participants might be concerned about the bi... \\
        \hline
        Risk Seeking & Significant 'Yes' position & The 41.5\% edge represents exactly the type of asymmetric opportunity that appeals to risk-seeking pa... \\
        \hline
        News Driven & Monitor transition announcements, potential 'Yes' position & This persona would focus on Trump's transition team appointments, early policy statements, and any p... \\
        \hline
        Macro Thinker & 'Yes' position as part of broader Trump policy basket & This market correlates with broader Trump administration policy implementation. Macro thinkers might... \\
        \hline
        Casual Participant & Small 'Yes' position & Simple thesis: 'Trump loves tariffs and acts fast.' The market seems to be betting he won't follow t... \\
        \hline
        Data Analyst & Cautious 'Yes' position pending deeper analysis & The 41.5\% gap demands quantitative validation. Data analysts would want to examine: Trump's first 10... \\
        \hline
\end{tabular}
\end{center}

%% ============================================
\section{Scenario Analysis}
%% ============================================

    \subsection*{Best Case}
    Trump announces comprehensive tariff strategy within first week, leveraging campaign momentum and prepared transition team. Multiple executive actions signed addressing China, EU, and Mexico with staggered effective dates throughout 2026. Legal infrastructure from previous term enables immediate implementation without extended review periods.

    \textbf{Probability Shift:} Yes probability increases to 85-90\% as multiple qualifying actions become highly likely

    \textbf{Key Triggers:}
    \begin{itemize}
            \item Early announcement of tariff-focused economic team
            \item Pre-inauguration meetings with trade advisors
            \item Campaign promise reinforcement in transition speeches
            \item Fast-track legal review completion
    \end{itemize}

    \subsection*{Worst Case}
    Transition period complications delay policy implementation. Legal review processes take longer than expected, economic advisors counsel delay due to market conditions, or Trump prioritizes other executive actions first. Initial tariff actions may lack specific effective dates or fail to meet technical qualification requirements.

    \textbf{Probability Shift:} Yes probability drops to 45-55\% as timing and technical execution become uncertain

    \textbf{Key Triggers:}
    \begin{itemize}
            \item Extended Cabinet confirmation delays
            \item Market volatility concerns raised by advisors
            \item Legal challenges to tariff authority
            \item Prioritization of immigration or other policy areas
    \end{itemize}

    \subsection*{Most Likely}
    Trump follows historical pattern of early tariff action but faces some procedural delays. At least one executive action on tariffs is signed in late January with effective date specified, likely targeting China or a specific trade dispute. Action meets technical requirements but may be narrower in scope than campaign rhetoric suggested.

    \textbf{Probability Shift:} Yes probability remains 70-80\% with single qualifying action highly probable

    \textbf{Key Triggers:}
    \begin{itemize}
            \item Confirmation of key trade officials by mid-January
            \item Routine legal review completion
            \item Standard transition team preparation
            \item Continuation of campaign trade messaging
    \end{itemize}


%% ============================================
\section*{Disclaimer}
%% ============================================

{\small\textit{This research report is for informational purposes only and does not constitute financial advice, investment advice, or a recommendation to buy or sell any securities or prediction market contracts. Prediction markets involve risk of loss. Past performance does not guarantee future results. Always do your own research and consider your own risk tolerance before participating in any market.}}

\end{document}
